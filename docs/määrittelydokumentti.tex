\documentclass[a4paper,12pt]{article}
\usepackage[utf8]{inputenc}
\usepackage{url}

\begin{document}
\title{TiRa labra - Määrittelydokumentti} 
\author{Jarmo Isotalo}
\maketitle
\begin{enumerate}
\item Toteutettavat algoritmit

Toteutan työssäni 3 kekoa, binäärikeon, binomikeon ja d-ary keon

\item Miksi valitsin nämä algoritmit

Kyseisten algoritmien valinnassa ei ollut mitään kummempia syitä. Kekojen toteutus vaikutti
kiinnostavalta, sillä se ei vaikuta liian hankalalta, ja soveltunee kohtuu hyvin toteutettavaksi rubyllä.

\item Mitä syötteitä ohjelma saa ja miten näitä käytetään\\
Syötteiden muoto tarkentuu myöhemmin. 
Niille voinee antaa syötteitä interaktiivisesta ja luultavasti parametreinä.

\item Tavoitteena olevat aika- ja tilavaativuudet (m.m. O-analyysi)

\begin{tabular}{|l|l|l|l|l|}
\hline
&Binary Heap & Binomial heap & D-ary \\\hline
Insert & $O (\log n)$ & $O (\log n)$ & ?\\\hline
Delete  & $O (\log n)$ & $O (\log n)$ & ?\\\hline
Build & $O (n \log n)$ & $O (\log n)$ & $O (n)$\\\hline
\end{tabular}

\item Lähteet\\
\url{https://en.wikipedia.org/wiki/Heap_(data_structure)}
\end{enumerate}

\end{document}
\documentclass[a4paper,12pt]{article}
\usepackage[utf8]{inputenc}
\usepackage{url}

\begin{document}
\title{TiRa labra - Määrittelydokumentti} 
\author{Jarmo Isotalo}
\maketitle

\section{Toteutettavat algoritmit}

Toteutan työssäni 3 kekoa, binäärikeon, binomikeon ja d-keon.
Aiheet muuttuivat kurssin aikana. Kuten toteutusdokumentissa totesin, totoeutun lopulta binäärikeon, kolmikeon ja d-keon.

\section{Miksi valitsin nämä algoritmit}

Kyseisten algoritmien valinnassa ei ollut mitään kummempia syitä. Kekojen toteutus vaikutti
kiinnostavalta, sillä se ei vaikuta liian hankalalta, ja soveltunee kohtuu hyvin toteutettavaksi rubyllä.

\section{Mitä syötteitä ohjelma saa ja miten näitä käytetään}
Syötteiden muoto tarkentuu myöhemmin. 
Niille voinee antaa syötteitä interaktiivisesta ja luultavasti parametreinä.

\section{Tavoitteena olevat aika- ja tilavaativuudet (m.m. O-analyysi)}

\subsection{Aikavaatimus}
\begin{tabular}{|l|l|l|l|l|}
\hline
&Binary Heap & Binomial heap & D-ary \\\hline
Insert & $O (\log n)$ & $O (\log n)$ & $O (\log n)$\\\hline
Delete  & $O (\log n)$ & $O (\log n)$ & $O (\log n)$\\\hline
Build & $O (n \log n)$ & $O (\log n)$ & $O (n)$\\\hline
\end{tabular}

\subsection{Tilavaatimus}
Tilavaativuuden arvoit eivät ole ihan hatusta heitettyjä, mutta ne riippuvat hyvin pitkälti toteutuksesta, joten rohkenen jättää ne melko avoimiksi. Ne toki tarkentuvat kun menille selviää kekojen  ja niiden metodien toteutustapa\\
\begin{tabular}{|l|l|l|l|l|}
\hline
&Binary Heap & Binomial heap & D-ary \\\hline
Insert & $O (\log n)$ & $O (\log n)$ & $O (\log n)$\\\hline
Delete  & $O (\log n)$ & $O (\log n)$ & $O (\log n)$\\\hline
Build & $O (\log n)$ & $O (\log n)$ & $O (n)$\\\hline
\end{tabular}


\section{Lähteet}
\url{https://en.wikipedia.org/wiki/Heap_(data_structure)}

\end{document}
\documentclass[a4paper,12pt]{article}
\usepackage[utf8]{inputenc}
\usepackage{url}

\begin{document}
\title{TiRa labra - Toteutusdokumentti} 
\author{Jarmo Isotalo}
\maketitle

\section{Toteutettavat algoritmit}

Toteutan työssäni 3 kekoa, binäärikeon, kolmikeon ja d-keon.

\section{Toteutuneet aika- ja tilavaativuudet (O-analyysi)}
\subsection{Aikavaatimus}
Tarkastelen tässä vain muutaman eri tapauksen aikavaativuuksia:
\begin{enumerate}
\item Keon alustaminen - Create\\
Jokaisessa keossa, binääri-,kolmi-, ja d-keossa operaatio on kutakuinkin saman kestoinen:
\begin{enumerate}
\item Aluksi alustetaan taulukko ja tallennetaan tietoon kunkin lapsien määrä.
Binäärikeossa lapsia on kaksi, kolmikeossa kolme ja d-keossa d kappaletta. $O(1)$
\item Koska Create tehdään tyhjälle keolle, on operaatio vakioaikainen. Tässä asetetaan taulun ensimmäiseen indeksiin parametrina saatu arvo. $O(1)$
\end{enumerate}
\item Kekoon lisääminen - Insert\\
Jokaisessa keossa, binääri-,kolmi-, ja d-keossa operaatio on kutakuinkin saman kestoinen:
\begin{enumerate}
\item Aluksi parametrina saatu elementti lisätään keon viimeiseen indeksiin.$O(1)$
\item Sitten indeksille suoritetaan \emph{heapify\_up}, joka siirtää elementtiä ylöspäin, kunnes keko noudattaa taas kekoehtoa. Tässä oletetaan, että keko noudatti kekoehtoa ennen elementin lisäämistä.
Tätä tapahtuu keon korkeuden verran. Eli insertin aikavaativuus on toteutuksessani $O(\log n)$
\end{enumerate}
\item Keosta poistaminen - Delete\\
Jokaisessa keossa, binääri-,kolmi-, ja d-keossa operaatio on kutakuinkin saman kestoinen:
\begin{enumerate}
\item Elementtiä keosta poistettaessa poistetaan elementti keon taulukon indeksistä 0. $O(1)$
\item Sen jälkeen siirretään keossa viimeisenä oleva elementti kekotaulukon indeksiin nolla. $O(1)$
\item Sitten kutsutaan \emph{heapify\_down} äsekettäin indeksiin nolla siirretylle, kunnes kekoehto toteutuu. $O(\log n)$. Lisäksi \emph{heapify\_down} tarkastaa onko elementillä suurempia lapsia. Tämän aikavaatimuus on 
\end{enumerate}
\end{enumerate}
Kekojen toteutuken vuoksi O notaation ajat ovat samankaltaisia, mutta todellisuudessa lasten määrän lisääminen nopeuttaa keon toimintaa. Kunnolliset BenchMarkit tulossa TODO\\\\
\begin{tabular}{|l|l|l|l|l|}
\hline
&Binary Heap & Three Heap & D-ary Heap \\\hline
Create & $O (1)$ & $O (1)$ & $O (1)$\\\hline
Insert & $O (\log n)$ & $O (\log n)$ & $O (\log n)$\\\hline
Delete  & $O (\log n)$ & $O (\log n)$ & $O (\log n)$\\\hline

\end{tabular}

\subsection{Tilavaatimus}
Tarkastelen tässä vain niiden metodien tilavaatimuksia, joiden aikavaatimuudet yllä1:
\begin{enumerate}
\item Keon alustaminen - Create\\
Jokaisessa keossa, binääri-,kolmi-, ja d-keossa operaatio on kutakuinkin saman kestoinen:
\begin{enumerate}
\item Aluksi alustetaan taulukko ja tallennetaan tietoon kunkin lapsien määrä.
Binäärikeossa lapsia on kaksi, kolmikeossa kolme ja d-keossa d kappaletta. $O(1)$
\item Koska Create tehdään tyhjälle keolle, on operaatio vakiotilainen. Tässä asetetaan taulun ensimmäiseen indeksiin parametrina saatu arvo. $O(1)$. Apumuuttujia ei tarvita.
\end{enumerate}
\item Kekoon lisääminen - Insert\\
Jokaisessa keossa, binääri-,kolmi-, ja d-keossa operaatio on kutakuinkin saman kestoinen:
\begin{enumerate}
\item Aluksi parametrina saatu elementti lisätään keon viimeiseen indeksiin.$O(1)$
\item Sitten indeksille suoritetaan \emph{heapify\_up}, joka siirtää elementtiä ylöspäin, kunnes keko noudattaa taas kekoehtoa. Tässä oletetaan, että keko noudatti kekoehtoa ennen elementin lisäämistä. \emph{heapify\_up} -metodia kutsutaan siis rekursiivisesti.
Tätä tapahtuu keon korkeuden verran. Insertin tilavaatimus on rekursiopinon kokoinen eli $O(\log n)$
\end{enumerate}
\item Keosta poistaminen - Delete\\
Jokaisessa keossa, binääri-,kolmi-, ja d-keossa operaatio on kutakuinkin saman kestoinen:
\begin{enumerate}
\item Elementtiä keosta poistettaessa poistetaan elementti keon taulukon indeksistä 0. $O(1)$
\item Sen jälkeen siirretään keossa viimeisenä oleva elementti kekotaulukon indeksiin nolla. $O(1)$
\item Sitten kutsutaan \emph{heapify\_down} äsekettäin indeksiin nolla siirretylle, kunnes kekoehto toteutuu. \emph{heapify\_down} -metodia kutsutaan rekursiiviseti eli sen tilavaatimus on rekursiopinon kokoinen $O(\log n)$ Lisäksi \emph{heapify\_down} tarkastaa onko elementillä suurempia lapsia. Tämän tilavaativuus on $O(d)$, jossa $d$ on lasten määrä eli vakio. tilavaatius on
\end{enumerate}
\end{enumerate}
\begin{tabular}{|l|l|l|l|l|}
\hline
&Binary Heap & Three Heap & D-ary Heap \\\hline
Create & $O (1)$ & $O (1)$ & $O (1)$\\\hline
Insert & $O (\log n)$ & $O (\log n)$ & $O (\log n)$\\\hline
Delete  & $O (\log n)$ & $O (\log n)$ & $O (\log n)$\\\hline

\end{tabular}

\section{Tiedostojen siainnit}
Tiedostojen polut on annettu suhteellisesti git repositorion juuresta katsottuna.
\begin{itemize}
\item Dokumentaatio
\begin{itemize}
\item \emph{docs/} 
\begin{itemize}
\item Suunnitteludokumentti
\item Toteutusdokumentti
\item Käyttöohje
\item Testausdokumentti
\end{itemize}
\end{itemize}

\item Lähdekoodi
\begin{itemize}
\item \emph{heaps/Heap/src/} 
\begin{itemize}
\item Binäärikeko - binary\_heap.rb
\item 3-keko - three\_heap.rb
\item D-keko - d\_heap.rb
\item Pino - stack.rb
\end{itemize}
\end{itemize}

\item Testit
\begin{itemize}
\item \emph{heaps/Heap/spec/} 
\begin{itemize}
\item Binäärikeon testit - binary\_heap\_spec.rb
\item 3-keon testit - three\_heap\_spec.rb
\item D-keon testit - d\_heap\_spec.rb
\item Pinon testit - stack\_spec.rb
\end{itemize}
\end{itemize}

\item Testien kattavuus
\begin{itemize}
\item \emph{heaps/Heap/src/doc} 
\begin{itemize}
\item index.html - Kaunis html pohjainen esitys koodin kattavuudesta
\end{itemize}
\end{itemize}

\end{itemize}


\section{Lähteet}
\begin{itemize}

\item Keot yleisesti\\
\url{https://en.wikipedia.org/wiki/Heap_(data_structure)}\\
\url{http://www.cs.helsinki.fi/u/tapasane/keot.pdf}\\
\url{http://www.cs.helsinki.fi/u/floreen/tira2012/tira.pdf}

\item Binäärikeko\\
\url{http://en.wikipedia.org/wiki/Binary_heap}

\item 3-keko\\
\url{http://www.cs.helsinki.fi/u/floreen/tira2012/teht07.pdf}

\item D-keko\\
\url{http://en.wikipedia.org/wiki/D-ary_heap}
\end{itemize}

\end{document}

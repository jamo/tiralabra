\documentclass[a4paper,12pt]{article}
\usepackage[utf8]{inputenc}
\usepackage{url}

\begin{document}
\title{TiRa labra - Toteutusdokumentti} 
\author{Jarmo Isotalo}
\maketitle

\section{Toteutettavat algoritmit}

Toteutan työssäni 3 kekoa, binäärikeon, binomikeon ja d-ary keon

\section{Toteutuneet aika- ja tilavaativuudet (O-analyysi)}
\subsection{Aikavaatimus}
Tarkastelen tässä vain muutaman eri tapauksen aikavaativuuksia:
\begin{enumerate}
\item Keon alustaminen - Create\\
Jokaisessa keossa, binääri-,kolmi-, ja d-keossa operaatio on kutakuinkin saman kestoinen:
\begin{enumerate}
\item Aluksi alustetaan taulukko ja tallennetaan tietoon kunkin lapsien määrä.
Binäärikeossa lapsia on kaksi, kolmikeossa kolme ja d-keossa d kappaletta. $O(1)$
\item Koska Create tehdään tyhjälle keolle, on operaatio vakioaikainen. Tässä asetetaan taulun ensimmäiseen indeksiin parametrina saatu arvo. $O(1)$
\end{enumerate}
\item Kekoon lisääminen - Insert\\
Jokaisessa keossa, binääri-,kolmi-, ja d-keossa operaatio on kutakuinkin saman kestoinen:
\begin{enumerate}
\item Aluksi parametrina saatu elementti lisätään keon viimeiseen indeksiin.$O(1)$
\item Sitten indeksille suoritetaan \emph{heapify\_up}, joka siirtää elementtiä ylöspäin, kunnes keko noudattaa taas kekoehtoa. Tässä oletetaan, että keko noudatti kekoehtoa ennen elementin lisäämistä.
Tätä tapahtuu keon korkeuden verran. Eli insertin aikavaativuus on toteutuksessani $O(\log n)$
\end{enumerate}
\item Keosta poistaminen - Delete\\
Jokaisessa keossa, binääri-,kolmi-, ja d-keossa operaatio on kutakuinkin saman kestoinen:
\begin{enumerate}
\item Elementtiä keosta poistettaessa poistetaan elementti keon taulukon indeksistä 0. $O(1)$
\item Sen jälkeen siirretään keossa viimeisenä oleva elementti kekotaulukon indeksiin nolla. $O(1)$
\item Sitten kutsutaan \emph{heapify\_down} äsekettäin indeksiin nolla siirretylle, kunnes kekoehto toteutuu. $O(\log n)$
\end{enumerate}
\end{enumerate}
Kekojen toteutuken vuoksi O notaation ajat ovat samankaltaisia, mutta todellisuudessa lasten määrän lisääminen nopeuttaa keon toimintaa. Kunnolliset BenchMarkit tulossa TODO\\\\
\begin{tabular}{|l|l|l|l|l|}
\hline
&Binary Heap & Three Heap & D-ary Heap \\\hline
Create & $O (1)$ & $O (1)$ & $O (1)$\\\hline
Insert & $O (\log n)$ & $O (\log n)$ & $O (\log n)$\\\hline
Delete  & $O (\log n)$ & $O (\log n)$ & $O (\log n)$\\\hline

\end{tabular}

\subsection{Tilavaatimus}
Tilavaativuuden arvoit eivät ole ihan hatusta heitettyjä, mutta ne riippuvat hyvin pitkälti toteutuksesta, joten rohkenen jättää ne melko avoimiksi. Ne toki tarkentuvat kun menille selviää kekojen  ja niiden metodien toteutustapa\\
\begin{tabular}{|l|l|l|l|l|}
\hline
&Binary Heap & Binomial heap & D-ary \\\hline
Insert & $O (\log n)$ & $O (\log n)$ & ?\\\hline
Delete  & $O (\log n)$ & $O (\log n)$ & ?\\\hline
Build & $O (\log n)$ & $O (\log n)$ & $O (n)$\\\hline
\end{tabular}


\section{Lähteet}
\url{https://en.wikipedia.org/wiki/Heap_(data_structure)}
\\Binäärikeko ja D-keko\\
\url{http://en.wikipedia.org/wiki/Binary_heap}
\url{http://en.wikipedia.org/wiki/D-ary_heap}
\url{http://www.cs.helsinki.fi/u/tapasane/keot.pdf}
\url{http://www.cs.unc.edu/~plaisted/comp750/05-binheaps.ppt}
\\Binomikeko:\\
\url{http://cs.anu.edu.au/people/Warren.Armstrong/apac/trunk/module2/binomial_heaps.pdf}
\url{https://www.cse.yorku.ca/~aaw/Sotirios/BinomialHeapAlgorithm.html}
\url{http://en.wikipedia.org/wiki/Binomial_heap}
\url{}

\end{document}

\documentclass[a4paper,12pt]{article}
\usepackage[utf8]{inputenc}
\usepackage{url}
\setlength{\parindent}{0.0cm}


\begin{document}
\title{TiRa labra - Testausdokumentti} 
\author{Jarmo Isotalo}
\maketitle

\section{Testausdokumentti}

\subsection{Mitä on testattu?}
Testit testaavat irrallisesti oleellisimmat (d-)keon metodit. Lisäksi testaavat kekojen järjestykseen liittyviä ominaisuuksia. Eli testit lisäävät kekoon N kpl Satunnaisia arvoja väliltä 0-1000 ja poistavat ne sen jälkeen. Testit tarkistavat että arvot tulevat suuruusjärjestyksessä.\\

Testit on toteutettu rspec-ohjelmalla ajettaviksi.

\subsection{Testien ajaminen}
Testit voidaan suorittaa kansiossa  \emph{heap/Heap/} suorittamalla komennon \emph{rspec spec/}. Tämä ajaa kaikki testit ja raportoi mahdollisista virheistä komentorivi käyttöliittymäänsä.
Testien ajaminen päivittää joka kerralla koodin testien kattavuuden laskennan. 

Testien kattavuuden näkee tiedostosta: 
\emph{heaps/Heap/src/doc/index.html} 


\subsection{Benchmark}
Kekojen toimintaa voi benchmarkata suorittamalla \emph{heaps/Heap/benchmark/} kansiossa olevan \emph{my\_benchmark\_spec.rb}  tiedoston.

Esimerkiksi kansiossa \emph{heaps/Heap/} ollessa komentorivillä \emph{rspec benchmark/}

Tämä tulostaa seuraavanlaisen kaavion:

\begin{verbatim}
                            user     system      total        real
Binaarikeko 100:lla     0.010000   0.000000   0.010000 (  0.004801)
Kolmikeko 100:lla       0.010000   0.000000   0.010000 (  0.003847)
D-keko 10 100:lla       0.000000   0.000000   0.000000 (  0.003700)
Binaarikeko 30000:lla   5.010000   0.010000   5.020000 (  5.026139)
Kolmikeko 30000:lla     3.820000   0.010000   3.830000 (  3.834460)
D-keko 10 30000:lla     3.450000   0.030000   3.480000 (  3.484981)
\end{verbatim} 
Tässä testissä D-keossa D oli 10.

Testit on ajettu i7 mac book prolla.

\subsection{Dokumentaation generointi}
rdoc dokumentaation saa generoituta komennolla \emph{rdoc --main d\_heap.rb} kansiossa  \emph{heaps/Heap/src/}.

\end{document}

\documentclass[a4paper,12pt]{article}
\usepackage[utf8]{inputenc}
\usepackage{url}
\setlength{\parindent}{0.0cm}


\begin{document}
\title{TiRa labra - Käyttöohje} 
\author{Jarmo Isotalo}
\maketitle

\section{Käyttöohje}
\subsection{Dokumentaation generointi}
Rdoc dokumentaation saa generoituta komennolla \emph{rdoc --main d\_heap.rb} kansiossa  \emph{heaps/Heap/src/}.

\subsection{Mitä keot syövät}
Kekoille voi syöttää mitä tahansa, mitä voi vertailla keskenään. Testeissä kekoihin lisätään ainoastaan numeroita, mutta mikään koodissa ei estä esim. kirjainten lisäämistä kekoon.


\subsection{Kekojen käyttäminen}
Jokainen keko tarjotaa saman käytettävyyden, ainut ero on keon lasten määrä sekä se, että d-keko vaatii lasten määrän konstruktorissaan eli DHeap.new kutsussa.\\


Hepify ja insert metodit ottavat parametrikseen n kappaletta kekoon syötettäviä tuotteita.\\

Metodien tarkempi toimivuus on kuvattu rdoc-dokumentaatiossa \emph{heaps/Heap/src/doc/index.html}.

\subsection{Keon toiminnan testaaminen}
Keot on toteutettu käytettäväksi osana jotain sovellusta, joten niiden käyttämiseen ei ole sen kummempaa käyttöliittymää rakennettu.\\

Voit testata keon käyttöä suorittamalla \emph{heaps/Heap/src/test.rb} Tällä pääset interaktiiviseen ruby shelliin jossa voit käyttää kaikkia rubyn ominaisuuksia ja näitä kolmea kekoa jotka toteutin. Riippuen laitteiston konfiguraatiosta voit joutua komentamaan  \emph{ruby heaps/Heap/src/test.rb}  \\

Alla esimerkkisessio:
\begin{verbatim}
jamo ~/g/tiralabra/heaps/Heap/src $ruby test.rb 
1.9.3-p125 :001 > d = DHeap.new 5
 => #<DHeap:0x007fbebd8095e8 @heap=[], @d=5> 
1.9.3-p125 :002 > d.insert 5,6,7,4,2,2,334,5,-123
 => [5, 6, 7, 4, 2, 2, 334, 5, -123] 
1.9.3-p125 :003 > d.remove_max
 => 334 
1.9.3-p125 :004 > d.remove_max
 => 7 
1.9.3-p125 :005 > d
 => #<DHeap:0x007fbebd8095e8 @heap=[6, 5, 5, 4, 2, 2, -123], @d=5> 
1.9.3-p125 :008 > d.inc_key 100, 0
 => nil 
1.9.3-p125 :009 > d
 => #<DHeap:0x007fbebd8095e8 @heap=[100, 5, 5, 4, 2, 2, -123], @d=5> 
1.9.3-p125 :010 > exit
jamo ~/g/tiralabra/heaps/Heap/src $
\end{verbatim}
Testauksen saa lopetettua komennolla \emph{exit}

\subsection{Tiedostojen sijainnit}
Kerrottu toteutusdokumentissa.

\subsection{Sovellus ei toimi kunnolla/koneessani on väärä ruby versio}
Oikean rubyversion asentaminen omalle koneelleen on helppoa, eikä se vaadi edes sen kummempia käyttöoikeuksia. \url{https://rvm.io/rvm/install/} \\
En kuitenkaan käsittele tässä sen tarkemmin sen asentamista.
\subsection{Ohjelman ajaminen laitoksen koneilla}
Ohjelma vaatii vähintään ruby 1.9 version, jota laitoksen koneilla ei suoraan ole. Sen saa kuitenkin helposti asennettua (ilman root oikeuksia) käyttämällä RVM:ää
\url{https://rvm.io/rvm/install/}
\emph{\\curl -L https://get.rvm.io | bash -s stable --ruby}
Yllä olevan komennon pitäisi asentaa ruby 1.9.3p194, jolla ohjelma toimii.
Joudut kuitenkin kertomaan kometoriville että se käyttää rvm:n kautta asennettua rubyä, etkä koneella valmiiksi olevaa. \emph{rvm use 1.9.3-p194} 
\\
Testien ajaminen vaatii gemin rspec asentamisen \emph{gem install rspec}
\end{document}
